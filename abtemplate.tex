\documentclass{ouab}
% Options:
%   oneassignment - mode that modifies table of contents for
%                  ABs containing only one assignment.
%   markcheck    - experimental version that checks whether part-marks
%                  for a question add up to the stated marks for the
%                  question.
% e.g. use \documentclass[oneassignment]{ouab}
%%%%%%%%%%%%%%%%%%%%%%%%%%%%%%%%%%%%%%%%%%%
% MODULE/DATE/TIME INFORMATION
%%%%%%%%%%%%%%%%%%%%%%%%%%%%%%%%%%%%%%%%%%%
\faculty{Faculty of STEM}
\modulecode{MXXX}
\moduletitle{Title of Module}
\abtitle{Title of AB}
%\absubtitle{Optional subtitle} %optional subtitle
\abyear{2015B}
\copyrightyear{2012} % If different from \abyear
\suppno{WEB 12345 6} %can be omitted for "DRAFT"
\versionno{1.0} %optional
%\optiontext{} %Can be used to change text appearing at top of multi-choice options.
%%%%%%%%%%%%%%%%%%%%%%%%%%%%%%%%%%%%%%%%%%%%
% INSTRUCTIONS ON FIRST PAGE
%%%%%%%%%%%%%%%%%%%%%%%%%%%%%%%%%%%%%%%%%%%%
\abinstructions{%
Module-specific instructions for the front page.
}% ends \abinstructions

%%%%%%%%%%%%%%%%%%%%%%%%%%%%%%%%%%%%%%%%%%%
\begin{document}
\maketitle

%%%%%%%%%%%%%%%%%%%%%%%%%%%%%%%%%%%
% FORMAT TMA/CMA/generic HEADERS
% \tma[cut off date]{number}[second line in contents]
% \cma[cut off date]{number}[second line in contents]
%
% Items in square brackets are optional
%
% Generic command also available:
% \assignment{NAME}[cut off date]{number}[second line in contents]
%
% where NAME is the type of assignment (TMA/CMA/EMA etc).
%%%%%%%%%%%%%%%%%%%%%%%%%%%%%%%%%%%%
% FORMAT FOR QUESTIONS
% \question[description]{marks}
%
% [description] is optional
%
% Use enumerate for subquestions. 
% enumitem is loaded, so you can do
% \begin{enumerate}[(i)] if you want to change the numbering system.
% Default numbering systems are:
% 1st level: (a), (b), (c), ...
% 2nd level: (i), (ii), (iii), ...
%%%%%%%%%%%%%%%%%%%%%%%%%%%%%%%%%%%%
% MULTI-CHOICE OPTIONS
% Two types available:
% \begin{options}
%   \item ...
%   \item ...
% \end{options}
%
% \begin{inlineoptions}[line spacing]{no_of_cols}
%   \item ...
%   \item ...
% \end{inlineoptions}

%%%%%%%%%%%%%%%%%%%%%%%%%%%%
\tma[12 November 2014]{01}[Covers units A and B]



\question[(essay question)]{25}
\emph{This question is about nothing in particular.}
\begin{enumerate}
\item hello\marks 4
\item goodbye running a long time until it goes over to a second line perhaps
\item 
\begin{enumerate}
\item Sub sub\marks 2
\item Sub sub question that runs a long time until it goes over to a second line\marks 2
\end{enumerate}
\end{enumerate}

\question{25}
\emph{This question is about nothing in particular.}
\begin{enumerate}
\item Marks can also be called using \mk 4
\item goodbye running a long time until it goes over to a second line perhaps
\item 
\begin{enumerate}
\item Sub sub\mk 2
\item Sub sub question that runs a long time until it goes over to a second line\mk 2
\end{enumerate}
\end{enumerate}
Emulate the OUTeX intertext command using the (built-in) enumitem command.
\begin{enumerate}[resume]
\item Numbering continues
\end{enumerate}


%%%%%%%%%%%%%%%%%%%%%%%%%%%%%
\cma{41}[Covers Units C and D]

\question[(inline multi-choice)]{25}
Choose an inline option.

\begin{inlineoptions}{5} %parameter specifies number of columns
\item yes
\item no
\item maybe
\item well...
\item never!
\item yes
\item yes
\item yes
\noitem %HACK. adds "blank" space so that the items line up as expected
\noitem
\end{inlineoptions}

\question[(inline multi-choice)]{25}
Choose an inline option from four columns.

\begin{inlineoptions}{4} %parameter specifies number of columns
\item yes
\item no
\item maybe
\item well...
\item never!
\item yes
\item yes
\item yes
\end{inlineoptions}

\question[(inline multi-choice)]{25}
Choose an inline option from three columns.

\begin{inlineoptions}{3} %parameter specifies number of columns
\item yes
\item no
\item maybe
\item well...
\item never!
\item yes
\end{inlineoptions}

\question
Choose an inline option at 1.5 spacing.
\begin{inlineoptions}[1.5]{3} %optional parameter specifies line spacing
\item yes
\item no
\item maybe
\item well...
\item never!
\item yes
\item yes
\item yes
\noitem %HACK. adds "blank" space so that the items line up as expected
\end{inlineoptions}

\question[(stupid question)]{25}
Choose an option.
\begin{options}
\item There are no options
\item There is one option
\item This option is the wrong choice
\end{options}


%%%%%%%%%%%%%%%%%%%%%%%%%%%%%%%%%%%%%
\assignment{QUIZ}[1 April 2015]{01}[Custom assessment type]

\question[(changing subquestion labels)]{8}
\begin{enumerate}[(i)]
\item enumitem lets you change the labelling of subquestions\marks 4
\item goodbye running a long time until it goes over to a second line perhaps
\end{enumerate}

\question[(changing subquestion labels 2)]{8}
\begin{enumerate}[(A)]
\item enumitem lets you change the labelling of subquestions\marks 4
\item goodbye running a long time until it goes over to a second line perhaps
\end{enumerate}

\question*[(Starred version showing no marks)]
This question has no marks associated with it on its first line.\mk 4

\question*
Another starred question, this time with no optional argument.

%%%%%%%%%%%%%%%%%%%%%%%%%%%%%
% Other ways to create assessments with/without optional parameters, 
% to show how they render on the page.

% Assignment with no cutoff or subtitle
\tma{02}

% Assignment with cutoff but no subtitle
\cma[an hour ago]{42}
\end{document}

