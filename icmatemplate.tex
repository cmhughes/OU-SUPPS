%% Use exactly ONE of the following four lines:
\documentclass{ouicma} 

%%%%%%%%%%%%%%%%%%%%%%%%%%%%%%%%%%%%%%%%%%%
% MODULE/DATE/TIME INFORMATION
%%%%%%%%%%%%%%%%%%%%%%%%%%%%%%%%%%%%%%%%%%%
\faculty{Faculty of Science, Technology, Engineering and Mathematics}
\modulecode{MXXX}
\moduletitle{Title of Module}
\icmayear{2015}
\icmatitle{Title of iCMA}
\suppno{WEB 12345 6} %can be omitted for "DRAFT"
\cutoffdate{26 November 2014}
\coversmaterialin{Units 1,2 and 3.}
%\optiontext{} %Can be used to change text appearing at top of multi-choice options.
%%%%%%%%%%%%%%%%%%%%%%%%%%%%%%%%%%%%%%%%%%%%
% INSTRUCTIONS ON FIRST PAGE
%%%%%%%%%%%%%%%%%%%%%%%%%%%%%%%%%%%%%%%%%%%%
\instructions{%
You should be able to answer all questions in iCMA~41 after studying 
Units 1, 2 and 3. There are 15 questions in total. Questions 1 to 5 relate 
to Unit 1, Questions 6 to 10 relate to Unit 2 and Questions 11 to 15 relate 
to Unit 3. Before attempting the questions for a particular unit, it may 
help to work through the corresponding practice quiz first.
 
The whole iCMA is marked out of 120. Every question is worth 8 marks. 
You are advised not to use Minitab to answer the questions. You may 
need to use a calculator to answer some questions. 

The completed iCMA should be returned to
\begin{quotation}
M140 Module Team\\
Department of Mathematics and Statistics\\
Faculty of Mathematics, Computing and Technology\\
The Open University\\
Walton Hall\\
Milton Keynes\\
MK7 6AA
\end{quotation}
The completed iCMA \emph{must arrive} by the cut-off date. 

You are advised to obtain proof of posting and to keep a copy of your 
answers for your own records. 

Note that the questions in this document are designed for on screen usage 
via the module website. Some of the instructions within questions may not 
quite suit this printed version, e.g. when the question refers to `box below' 
this should be interpreted as `box in margin'. 
}% ends \examinstructions

% \declaration{} Optional: you can use this to write your own declaration page.
%%%%%%%%%%%%%%%%%%%%%%%%%%%%%%%%%%%%%%%%%%%
\begin{document}
\pagenumbering{arabic}
\maketitle
%\end{document}

%%%%%%%%%%%%%%%%%%%%%%%%%%%%%%%%%%%
% Begin writing document below here
%%%%%%%%%%%%%%%%%%%%%%%%%%%%%%%%%%%

\question
Colour in all the boxes. Don't go over the edges.

\begin{enumerate}
\item hello \bigskip\answerbox
\item Here's a wider one.\bigskip \wideanswerbox
\item answer the subquestions
\begin{enumerate}
\item yo\medskip \answerbox 
\item yaasdasdas asdjk asdfjk jkl sjkdfjkl  running a long time until it goes over to a second line perhaps\wideanswerbox
\end{enumerate}
\end{enumerate}


\question

Choose an option

\begin{inlineoptions}{5}%parameter specifies number of columns
\item yes
\item no
\item maybe
\item well...
\item never!
\item yes
\item yes
\item yes
\item yes
\item yes\answerbox
\end{inlineoptions}

\question
You can have non-inline options too:
\begin{options}
\item There are no options
\item There is one option
\item This option is the wrong choice\answerbox
\end{options}

\question

Choose an option

\begin{inlineoptions}{5}%parameter specifies number of columns
\item yes
\item no
\item maybe
\item well...
\item never!
\item yes
\item yes
\item yes
\item yes
\item yes\answerbox
\end{inlineoptions}

\question
Choose an inline option at 1.5 spacing.
\begin{inlineoptions}[1.5]{3} %optional parameter specifies line spacing
\item yes
\item no
\item maybe
\item well...
\item never!
\item yes
\item yes
\item yes
\noitem %HACK. adds "blank" space so that the items line up as expected
\end{inlineoptions}

\question
You can have non-inline options too:
\begin{options}
\item There are no options
\item There is one option
\item This option is the wrong choice\answerbox
\end{options}


\question
You can have non-inline options too:
\begin{options}
\item There are no options
\item There is one option
\item This option is the wrong choice\answerbox
\end{options}

blah blah 
\end{document}