%
% OU Tikz figure styles.
%

% 21/12/16: Updated to reflect outikz v2.2, and improve style




% A reminder about tikz coordinates:
% 
% cartesian: (1,2) = "one times current x-vector, two times current y-vector"
%            (10pt,2cm) using units, but gets affected by "scale" settings
% polar: (angle in degrees : distance)
%
% Incremental:
% ++(1,2) move by (1,2) from previous positn, and make new position current


\documentclass[11pt,a4paper]{article}

\usepackage{outikz} % include the package, and OU style definitions

%
% page setup
%

\setlength{\textwidth}   {150mm}
\setlength{\textheight}  {270mm}
\setlength{\topmargin}     {-15mm}
\setlength{\headheight}    {0mm}
\setlength{\headsep}       {0mm}
\setlength{\oddsidemargin} {-10mm}
\setlength{\evensidemargin}{-10mm}

%
% Visual formatting
%

\setlength{\parskip}{2.5ex}   % space between paragraphs  was 10pt
\setlength{\parindent}{0em}
\pagestyle{plain}



\begin{document}




\section*{OU Tikz styles}

As defined in \texttt{outikz.sty}

\subsection*{Lines}
\vspace*{-\baselineskip}

{\renewcommand{\arraystretch}{1.4}
\begin{tabular}{p{0.5\textwidth}p{0.15\textwidth}p{0.25\textwidth}l}
& & \textbf{Dashed versions} \\

Default (1pt) \verb|\draw|   &  \tikz \draw (0,0) --++ (2,0); &
\verb|\draw[dashed]|         &   \tikz \draw[dashed] (0,0) --++ (2,0);\\

Thick (1.6pt) \verb|\draw[thick]| & \tikz \draw[thick] (0,0) --++ (2,0); &
\verb|\draw[thick,dashed]|   & \tikz \draw[thick,dashed] (0,0) --++ (2,0);\\

Construction lines (0.25pt) \verb|\draw[thin]| 
        &  \tikz \draw[thin] (0,0) --++ (2,0); &
\verb|\draw[thin,dashed]| & \tikz \draw[thin,dashed] (0,0) --++ (2,0);\\

Grid lines (0.5pt, 20\% black) \verb|\draw[grid]|
& \tikz \draw[grid] (0,0) -- ++(2,0);\\[5pt]

Arc (0.25pt) \verb|\draw[arc]| & \tikz \draw[arc] (0,0) arc(90:180:1);

\end{tabular}
}



\subsection*{Lines with arrows}

{\renewcommand{\arraystretch}{1.5}
\begin{tabular}{p{0.68\textwidth}l}
Axis \verb|\draw[axis]| & \tikz \draw[axis] (0,0) --++ (4,0);\\

Axis with ticks (0.5pt by 3pt)  and label \newline
\verb|\draw[axis] ... node[xlab]{$x$}| \newline
(Ticks need to be fixed length, independent of scaling: \newline
\verb|  \coordinate (t1) at (x,y);|\newline
\verb|  \draw[ticks, reset cm] (t1)--++(0,-\ticklength)|)
& \tikz {
  \draw[axis] (0,0) --++(4,0) node[xlab] {$x$};
  \coordinate (t1) at (0,0);
  \coordinate (t2) at (1,0);
  \coordinate (t3) at (2,0);
  \coordinate (t4) at (3,0);

  \draw[ticks, reset cm] (t1) --++(0,-\ticklength);
  \draw[ticks, reset cm] (t2) --++(0,-\ticklength);
  \draw[ticks, reset cm] (t3) --++(0,-\ticklength);
  \draw[ticks, reset cm] (t4) --++(0,-\ticklength);
}\\

Thick axis \verb|\draw[thickaxis]| \newline
(for screencasts etc only)
& \tikz \draw[thickaxis] (0,0) --++ (4,0);\\

Arc \verb|\draw[arc,-arcarrow]| & \tikz \draw[arc,-arcarrow] (0,0) arc(180:90:1);\\

Vector \verb|\draw[vector]| & \tikz \draw[vector] (0,0) --++ (2,0);\\
Force \verb|\draw[force]|  & \tikz \draw[force] (0,0) --++ (2,0);\\
Acceleration \verb|\draw[acceleration]| & \tikz \draw[acceleration] (0,0)--++ (2,0);\\

Mapping \verb|\draw[mapping]| & \tikz \draw[mapping] (0,0) --++ (2,0);\\
Network \verb|\draw[network]| & \tikz \draw[network] (0,0) --++ (2,0);\\
Dimension \verb|\draw[dimension]| & \tikz \draw[dimension] (0,0) --++ (2,0);\\
Dimension with markers \verb|\draw[dimensionmark]| 
  & \tikz \draw[dimensionmark] (0,0) --++ (2,0);\\
Compass North \verb|\draw[-compassarrow]| & 
\tikz \draw[-compassarrow] (0,0) --++(0,1);
\end{tabular}
}





\subsection*{Miscellaneous}
\vspace*{-1em}

{\renewcommand{\arraystretch}{1.5}
\begin{tabular}{p{0.9\textwidth}l}
Point marker  \newline
\verb|\draw (0,0) node[point, label=right:{$P$}]{};|
& \tikz \draw (0,0) node[point, label=right:{$P$}]{};\\

Square point \newline
\verb|\draw (0,0) node[sqrpoint, label=right:{$Q$}]{};|
& \tikz \draw (0,0) node[sqrpoint, label=right:{$Q$}]{};\\


Open/closed points \newline
\verb|\draw (0,0) node[open]{} -- (1,0) node[closed]{};| 
& \tikz \draw (0,0) node[open]{} -- ++(1,0) node[closed]{};\\

Tag/label \newline
\verb|\draw[thin] (0,0) -- (1,0) node[tag, fill=M337bluefill, anchor=west]{A label};|  \newline \null
& \tikz \draw[thin] (0,0) --++ (2,0) node[tag, fill=M337bluefill, anchor=west]{A label};\\



\vspace{-2\baselineskip}Cloud \newline
\verb|\draw (1,0) node[oucloud, fill=M337bluefill](cloudname){A label};|\newline
\verb|\draw[thin] (cloudname) edge (0,0);|
& \tikz 
{ \draw (2,0) node[oucloud, fill=M337bluefill](cloudname){A label};
\draw[thin] (cloudname) edge (0,0);
}\\

Spring \newline
\verb|\draw[MST210spring] (0,0) -- (2,0) node[pos=0.5,above=4]{$k,l_0$}  node[point]{\phantom{a}};|
& 
\tikz \draw[MST210spring] (0,0) --++(3,0)  node[pos=0.5, above=4] {$k,l_0$} node[point] {\phantom{a}};\\


Damper\newline
\verb|\draw[MST210damper] (0,0) -- (2,0) node[pos=0.5,above=4]{$r$};|
& \tikz \draw[MST210damper] (14,-8) --++ (3,0) node[pos=0.5, above=4]{$r$};

\end{tabular}
}









\subsection*{Colours}
\vspace*{-1em}


\subsubsection*{M337 line colours (for example)}
\vspace*{-1em}


{\renewcommand{\arraystretch}{1.5}
\begin{tabular}{p{0.6\textwidth}ccc}

& Line & Fill 100\% & Fill 40\% \\ \\

\vspace*{-3\baselineskip}
Blue\newline
\verb|\draw[M337blue]|\newline
\verb|\draw[M337blue,fill=M337blue]|\newline
\verb|\draw[M337bluefill,fill=M337bluefill]|
& \tikz \draw[M337blue] (0,0) --++ (2,0);
&\tikz \draw[M337blue, fill=M337blue] (0,0) rectangle ++(2,1);
& \tikz \draw[M337bluefill, fill=M337bluefill] (0,0) rectangle ++(2,1);\\ \\


\vspace*{-3\baselineskip}
Red\newline
\verb|\draw[M337red]|\newline
\verb|\draw[M337red,fill=M337red]|\newline
\verb|\draw[M337redfill,fill=M337redfill]|
& \tikz \draw[M337red] (0,0) --++ (2,0);
&\tikz \draw[M337red, fill=M337red] (0,0) rectangle ++(2,1);
& \tikz \draw[M337redfill, fill=M337redfill] (0,0) rectangle ++(2,1);\\ \\


\vspace*{-3\baselineskip}
Green\newline
\verb|\draw[M337green]|\newline
\verb|\draw[M337green,fill=M337green]|\newline
\verb|\draw[M337greenfill,fill=M337greenfill]|
& \tikz \draw[M337green] (0,0) --++ (2,0);
&\tikz \draw[M337green, fill=M337green] (0,0) rectangle ++(2,1);
& \tikz \draw[M337greenfill, fill=M337greenfill] (0,0) rectangle ++(2,1);\\ \\

\vspace*{-3\baselineskip}
Purple\newline
\verb|\draw[M337purple]|\newline
\verb|\draw[M337purple,fill=M337purple]|\newline
\verb|\draw[M337purplefill,fill=M337purplefill]|
& \tikz \draw[M337purple] (0,0) --++ (2,0);
&\tikz \draw[M337purple, fill=M337purple] (0,0) rectangle ++(2,1);
& \tikz \draw[M337purplefill, fill=M337purplefill] (0,0) rectangle ++(2,1);\\ \\

\vspace*{-3\baselineskip}
Orange\newline
\verb|\draw[M337orange]|\newline
\verb|\draw[M337orange,fill=M337orange]|\newline
\verb|\draw[M337orangefill,fill=M337orangefill]|
& \tikz \draw[M337orange] (0,0) --++ (2,0);
&\tikz \draw[M337orange, fill=M337orange] (0,0) rectangle ++(2,1);
& \tikz \draw[M337orangefill, fill=M337orangefill] (0,0) rectangle ++(2,1);\\ \\


\end{tabular}
}







\subsubsection*{Structured content colours}

\begin{tabular}{p{0.3\textwidth}p{0.25\textwidth}p{0.2\textwidth}p{0.25\textwidth}}

\vspace*{-3\baselineskip}
\texttt{figurebox}:\newline
(for figure backgrounds)
& \tikz \draw[figurebox, fill=figurebox] (0,0) rectangle ++(3,1.5);

&
\vspace*{-3\baselineskip}
\texttt{greenbox} & \tikz \draw[figurebox, fill=greenbox] (0,0) rectangle ++(3,1.5);\\ 


\vspace*{-3\baselineskip}
\texttt{buffbox} & \tikz \draw[figurebox, fill=buffbox] (0,0) rectangle ++(3,1.5);
&
\vspace*{-3\baselineskip}
\texttt{bluebox} & \tikz \draw[figurebox, fill=bluebox] (0,0) rectangle ++(3,1.5);


\end{tabular}





\vspace*{-2\baselineskip}
\subsection*{Figure containing boxes}
\vspace*{-\baselineskip}
(Drawn with buff background to improve visability!)\\
Of course, individuals may have their own preferred way of achieving the same effects.

\vspace*{-\baselineskip}
\subsubsection*{Single figures}
\vspace*{-\baselineskip}


\begin{verbatim}
\begin{outikzfig}[<colour>]{<width>}     <colour> = one of figurebox/bluebox/buffbox/greenbox 
\begin{tikzpicture}                      <width>  = one of figure/margin/solution/fullwidth
...
\end{tikzpicture}
\end{outikzfig}
\end{verbatim}


\begin{outikzfig}[buffbox]{figure}
\begin{tikzpicture}
\draw (0,0) node{Standard figure width = 360pt};
\end{tikzpicture}
\end{outikzfig}


\begin{outikzfig}[buffbox]{solution}
\begin{tikzpicture}
\draw (0,0) node{Solution figure width = 240pt};
\end{tikzpicture}
\end{outikzfig}
\hfill
\begin{outikzfig}[buffbox]{margin}
\begin{tikzpicture}
\draw (0,0) node{Margin figure width = 144pt};
\end{tikzpicture}
\end{outikzfig}

\begin{outikzfig}[buffbox]{fullwidth}
\begin{tikzpicture}
\draw (0,0) node{Full width figure = 516pt};
\end{tikzpicture}
\end{outikzfig}

\subsubsection*{Multiple figures}
\vspace*{-\baselineskip}
\begin{verbatim}
\begin{outikzmultifig}[<colour>]{<width>}{<number>}  % <colour> = as above
                                                     % <width>  = as above
                                                     % <number> = number columns
\ousubfig                             % use \ousubfignum to number the figures
\begin{tikzpicture}                      
...
\end{tikzpicture}

\ousubfig
\begin{tikzpicture}                      
...
\end{tikzpicture}

...
\end{outikzmultifig}
\end{verbatim}



\begin{outikzmultifig}[buffbox]{figure}{3}

\ousubfig
\begin{tikzpicture}[scale=0.9]
\draw[axis] (0,0) --++(3,0) node[xlab]{$x$};
\draw[axis] (0,0) --++(0,3) node[ylab]{$y$};
\draw[domain=0:3] plot (\x,\x);
\end{tikzpicture}


\ousubfig 
\begin{tikzpicture}[scale=0.9]
\draw[axis] (0,0) --++(3,0) node[xlab]{$x$};
\draw[axis] (0,0) --++(0,3) node[ylab]{$y$};
\draw[domain=0:{sqrt(3)}] plot (\x,\x*\x);
\end{tikzpicture}

\ousubfig
\begin{tikzpicture}[scale=0.9]
\draw[axis] (0,0) --++(3,0) node[xlab]{$x$};
\draw[axis] (0,0) --++(0,3) node[ylab]{$y$};
\draw[domain=0.4:3] plot (\x, 1/\x);
\end{tikzpicture}


\ousubfig 
\begin{tikzpicture}[scale=0.9]
\draw[axis] (0,0) --++(3,0) node[xlab]{$x$};
\draw[axis] (0,0) --++(0,3) node[ylab]{$y$};
\draw[domain=0:2.9] plot (\x, 3-\x);
\end{tikzpicture}


\ousubfig 
\begin{tikzpicture}[scale=0.9]
\draw[axis] (0,0) --++(3,0) node[xlab]{$x$};
\draw[axis] (0,0) --++(0,3) node[ylab]{$y$};
\draw[domain=0:{sqrt(2.5)}] plot (\x, 2.8-\x*\x);
\end{tikzpicture}



\ousubfig 
\begin{tikzpicture}[scale=0.9]
\draw[axis] (0,0) --++(3,0) node[xlab]{$x$};
\draw[axis] (0,0) --++(0,3) node[ylab]{$y$};
\draw[domain=0.4:3] plot (\x, 3-1/\x);
\end{tikzpicture}


\ousubfig 
\begin{tikzpicture}[scale=0.9]
\draw[axis] (0,0) --++(3,0) node[xlab]{$x$};
\draw[axis] (0,0) --++(0,3) node[ylab]{$y$};
\draw[domain=0:3] plot (\x, {1.5+1.5*sin(\x*180)} );
\end{tikzpicture}


\end{outikzmultifig}




\begin{outikzmultifig}[buffbox]{figure}{2}

\ousubfignum
\begin{tikzpicture}[xscale=1.5]
\draw[axis] (0,0) --++(3,0) node[xlab]{$x$};
\draw[axis] (0,0) --++(0,3) node[ylab]{$y$};
\draw[domain=0:3] plot (\x,\x);
\end{tikzpicture}

\ousubfignum 
\begin{tikzpicture}[xscale=1.5]
\draw[axis] (0,0) --++(3,0) node[xlab]{$x$};
\draw[axis] (0,0) --++(0,3) node[ylab]{$y$};
\draw[domain=0:{sqrt(3)}] plot (\x,\x*\x);
\end{tikzpicture}



\ousubfignum
\begin{tikzpicture}[xscale=1.5]
\draw[axis] (0,0) --++(3,0) node[xlab]{$x$};
\draw[axis] (0,0) --++(0,3) node[ylab]{$y$};
\draw[domain=0:3] plot (\x, {1.5+1.5*sin(\x*180)} );
\end{tikzpicture}



\end{outikzmultifig}



\end{document}


