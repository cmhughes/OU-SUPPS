\documentclass{outn}
% Options:
%   specsolns    - mode that produces output with marks, solnmarksplus,
%                  and remarks suppressed, and text in green.
% e.g. use \documentclass[specsolns]{outn}
%%%%%%%%%%%%%%%%%%%%%%%%%%%%%%%%%%%%%%%%%%%
% MODULE/DATE/TIME INFORMATION
%%%%%%%%%%%%%%%%%%%%%%%%%%%%%%%%%%%%%%%%%%%
\faculty{Faculty of Science, Technology, Engineering and Mathematics}
\modulecode{MXXX}
\moduletitle{Title of Module}
\tntitle{Title of TN}
\tnyear{2015B}
\copyrightyear{2012} % If different from \abyear
%\tutoronlytitle{ {\bfseries Title for tutor-only environment\par}}      % specifies title for 'Tutor-only' environment
%\studentonlytitle{Title for student-only environment\par}  % specifies title for 'Student-only' environment
%\specsolnscolor{blue} % can be used to change text colour in specsolns mode
%\optiontext{} %Can be used to change text appearing at top of multi-choice options.
\oulogofilebase{ou_greyscale_masterlogo_29mm}% Overrides internal OU logo specification
%\switchmarks % Can be used to place marks to right of comments
%% alternative way of updating the above information
%\metadataset{faculty=STEM, 
%            module title=Title of Module,
%            module code=MST142,
%            presentation=2017B,
%            copyright year=1999,
%            document title=Title of Tutor notes,
%            specimen solutions color=blue!50!black,
%            multiple choice option text=choices,
%            tutor only title={Title for tutor-only environment\par},
%            student only title=Student material only\par,
% 			 switch marks, % Does same as \switchmarks
%            }
%%%%%%%%%%%%%%%%%%%%%%%%%%%%%%%%%%%%%%%%%%%
\begin{document}
\maketitle

%%%%%%%%%%%%%%%%%%%%%%%%%%%%%%%%%%%
% FORMAT TMA/CMA/generic HEADERS
% \tma[cut off date]{number}[second line in contents]
% \cma[cut off date]{number}[second line in contents]
%
% Items in square brackets are optional
%
% Generic command also available:
% \assignment{NAME}[cut off date]{number}[second line in contents]
%
% where NAME is the type of assignment (TMA/CMA/EMA etc).
%%%%%%%%%%%%%%%%%%%%%%%%%%%%%%%%%%%%
% FORMAT FOR QUESTIONS
% \question[description]{marks}
%
% [description] is optional
%
% Use enumerate for subquestions. 
% enumitem is loaded, so you can do
% \begin{enumerate}[(i)] if you want to change the numbering system.
% Default numbering systems are:
% 1st level: (a), (b), (c), ...
% 2nd level: (i), (ii), (iii), ...
%%%%%%%%%%%%%%%%%%%%%%%%%%%%%%%%%%%%
% MULTI-CHOICE OPTIONS
% Two types available:
% \begin{options}
%   \item ...
%   \item ...
% \end{options}
%
% \begin{inlineoptions}{no_of_cols}
%   \item ...
%   \item ...
% \end{inlineoptions}

%%%%%%%%%%%%%%%%%%%%%%%%%%%%

% The syntax for defining assignments and questions 
% is identical to that used in the ouab class.
% Starred versions don't add a new page.
\tma[12 November 2014]{01}[Covers units A and B]

\question[(essay question)]{25}\label{myqnnum}
\remark{For making short comments for tutors.}

\begin{enumerate}
\item hello\mk 4
\item goodbye running a long time until it goes over to a second line perhaps
\item 
\begin{enumerate}
\item Sub sub\mk 2
\item Sub sub\mk 2
\end{enumerate}
\subtotal[for part \theenumi]{4}
\end{enumerate}
\textref{Unit A, pages 1--2}% Reference to course material
\total[for Question \thequestion]{5} % Total marks for a question


\begin{longremark}
You can place longer remarks inside the longremark environment.
\end{longremark}

\begin{references}
You can place references inside the references environment.
\end{references}

\question*\label{myqnnum-a}
% Put your solution inside the solution environment
% if you want to try using the studenttex option.
\begin{solution}
\begin{enumerate}
\item First part of the solution, inside the solution environment. \mk{4}
\item Some mathematics: $\int_0^\infty x$ and \[a^2=b^2=c^2\mk{2}\] 

\remark{Award no marks for this part of the question.}

\item Marks with marginal comments. \mk[Method. Ignore all attempts at this question.]{40}

\item Starred enumerate environment gives you inline enumerate:
\begin{enumerate*}
\item first part
\item second part
\item third part\mk{3}
\end{enumerate*}

\item The final part.\mk{3}

\subtotal[for part~\theenumi]{1}
\end{enumerate}
\textref{Unit B, Theorem 1.1}\total[for Question \ref{myqnnum-a}]{5}
\end{solution}

\question{25}
\begin{solution}

\begin{enumerate}
\item Marks can also be called using \mk 4
\item Formatting in marks: these don't get automatically summed.\mk[1A 2M]{\bf 3}
\item 
\begin{enumerate}
\item Sub sub\mk 2
\item Sub sub question that runs a long time until it goes over to a second line\mk 2
\end{enumerate}
\end{enumerate}
Emulate the OUTeX intertext command using the (built-in) enumitem command.
\begin{enumerate}[resume]
\item Numbering continues
\end{enumerate}
\end{solution}
\total{3}

\question{10}
\begin{enumerate}
  \item Subtotals can be calculated automatically using \verb!\subtotal*! 
    or \verb!\subtotal*[comment]! \mk 2 
  \item Totals can be calculated automatically using  \verb!\total*! 
    or \verb!\total*[comment]! \mk 3

\subtotal*
\item Marks that you don't want added to the subtotal/total counter can be inserted using \verb!\mk*! \mk* 2
\end{enumerate}
\subtotal*

\bigskip
\total*

\question{3}
\begin{enumerate}
  \item The lines after a \verb!subtotal! command can be customized using
    the \verb!\setSubtotalHline[<options>]! command, the \verb!<options>! 
    are \verb!moveleft!, \verb!width!, \verb!height!, \verb!color!, 
    and \verb!draw line!. 
    These options can be used in any combination (and should probably be set once in the preamble); for example, 
    \begin{itemize}
      \item default behaviour

        \subtotal*
      \item \verb!\setSubtotalHline[moveleft=0pt]!
        \setSubtotalHline[moveleft=0pt]

        \subtotal*
      \item \verb!\setSubtotalHline[moveleft=-2cm,width=.5\textwidth]!
        \setSubtotalHline[moveleft=-2cm,width=.5\textwidth]

        \subtotal*
      \item \verb!\setSubtotalHline[color=green]!
        \setSubtotalHline[color=green] \mk{2}

        \subtotal{3}
      \item \verb!\setSubtotalHline[moveleft=0pt,height=5pt,color=orange]!
        \setSubtotalHline[moveleft=0pt,height=5pt,color=orange]

        \subtotal*
      \item turn off: \verb!\setSubtotalHline[draw line=false]! \mk{1}
            \setSubtotalHline[draw line=false]
        
        \subtotal*
    \end{itemize}
  \item The lines after a \verb!total! command can be customised 
    using the \verb!\setTotalHline[<options>]! command, and have
    exactly the same options as its \verb!\setSubtotalHline! counterpart.

    For example,
    \begin{itemize}
      \item default behaviour
        
        \total{1}
      \item \verb!\setTotalHline[moveleft=10pt]!
        \setTotalHline[moveleft=10pt]

        \total*
      \item \verb!\setTotalHline[moveleft=2cm,width=.75\textwidth]!
        \setTotalHline[moveleft=2cm,width=.75\textwidth]

        \total*
      \item \verb!\setTotalHline[color=purple]!
        \setTotalHline[color=purple] 

        \total{3}
      \item \verb!\setTotalHline[moveleft=0pt,height=.5pt,color=yellow]!
        \setTotalHline[moveleft=0pt,height=.5pt,color=yellow]

        \total*
      \item turn off: \verb!\setTotalHline[draw line=false]!
            \setTotalHline[draw line=false]
        
        \total*
    \end{itemize}
  \item Horizontal rules can be drawn manually without using the \verb!\total! (and friends) command(s) using 
    the \verb!\ourule[<options>]! command, where the \verb!<options>! are 
    exactly the same options as those for \verb!\setSubtotalHline!. For example:
    \begin{itemize}
      \item \verb!\ourule!
    \ourule
      \item \verb!\ourule[color=orange]!
    \ourule[color=orange]
      \item \verb1\ourule[color=black!50!red,height=3pt,width=.5\textwidth]1
    \ourule[color=black!50!red,height=3pt,width=.5\textwidth]
  \item You can specify the style of the \verb!\ourule! command globally using, for example, \verb!\setOUrule[color=red,moveleft=-2cm,width=.75\textwidth]!

\setOUrule[color=red,moveleft=-2cm,width=.75\textwidth]

Now, when using simply \verb!\ourule!, the output is:

\ourule

And then you can overwrite certain aspects as you see fit, for example \verb!\ourule[color=blue]!

\ourule[color=blue]

    \end{itemize}
\end{enumerate}

\question{4}
\begin{enumerate}
  \item The \verb|\tutoronly| command \emph{only} produces output in the tutor version of the tutor notes, 
    and is ignored when the \verb|specsolns| option is used. For example
    \tutoronly{ {\color{blue} students will not see this!}}.
  \item The \verb|tutor| environment is only visible in tutor notes, 
    and is ignored when the \verb|specsolns| option is used. For example

    \begin{tutor}
      This environment will only be seen in the tutor notes.

      It will not be seen in the student version.
    \end{tutor}
    \tutoronly{\pagebreak}
  \item The \verb|\studentonly| command \emph{only} produces output in the 
    special solutions version of tutor notes (so \verb|specsolns| must be active); for example \studentonly{this is only output in student solutions}
  \item The \verb|student| environment is only visible in student solutions, 
    \studentonly{\pagebreak}
    when the \verb|specsolns| option is used. For example

    \begin{student}
      This environment will only be seen in the student solutions.

      It will not be seen in the tutor notes.
    \end{student}
  \item The \verb|\tutororstudent| takes two parameters, the first will only be seen in the tutor version (when \verb|specsolns|
    is not active), the second will only be seen by students in the student version (when \verb|specsolns| is active).
    For example: \tutororstudent{for tutors!}{for students!}
\end{enumerate}

%%%%%%%%%%%%%%%%%%%%%%%%%%%%%
\cma*{41}[Covers Units C and D]

\question[(inline multi-choice)]{25}
\remark{Inline options work in the Tutornote class file in exactly the same way as in the assignment booklet.}

\begin{inlineoptions}{5} %parameter specifies number of columns
\item yes
\item no
\item maybe
\item well...
\item never!
\item yes
\item yes
\item yes
\noitem %HACK. adds "blank" space so that the items line up as expected
\noitem
\end{inlineoptions}

\question[(inline multi-choice)]{25}
Choose an inline option from four columns.

\begin{inlineoptions}{4} %parameter specifies number of columns
\item yes
\item no
\item maybe
\item well...
\item never!
\item yes
\item yes
\item yes
\end{inlineoptions}

\question*
Choose an inline option at 1.5 spacing.
\begin{inlineoptions}[1.5]{3} %optional parameter specifies line spacing
\item yes
\item no
\item maybe
\item well...
\item never!
\item yes
\item yes
\item yes
\noitem %HACK. adds "blank" space so that the items line up as expected
\end{inlineoptions}

\question[(stupid question)]{25}
Choose an option.
\begin{options}
\item There are no options
\item There is one option
\item This option is the wrong choice
\end{options}


%%%%%%%%%%%%%%%%%%%%%%%%%%%%%%%%%%%%%
\assignment{QUIZ}[1 April 2015]{01}[Custom assessment type]
\samepageassignment{QUIZB}[1 April 2015]{01}[Custom assessment type]

\question[(changing subquestion labels)]{8}
\begin{enumerate}[(i)]
\item enumitem lets you change the labelling of subquestions\mk 4
\item goodbye running a long time until it goes over to a second line perhaps
\end{enumerate}

\question[(changing subquestion labels 2)]{8}
\begin{enumerate}[(A)]
\item enumitem lets you change the labelling of subquestions\mk 4
\item goodbye running a long time until it goes over to a second line perhaps
\end{enumerate}

\question*[(Starred version showing no marks)]
This question has no marks associated with it on its first line.\mk 4

\question*
Another starred question, this time with no optional argument.

%%%%%%%%%%%%%%%%%%%%%%%%%%%%%
% Other ways to create assessments with/without optional parameters, 
% to show how they render on the page.

% Assignment with no cutoff or subtitle
\tma{02}

% Assignment with cutoff but no subtitle
\cma[an hour ago]{42}
\end{document}

