\documentclass[remote,showsolutions]{ouexam} 
% Main available options:
%  remote           For remote exam papers
%  specimen			For specimen exam papers
%  secondspecimen	For 2nd specimen exam
%  showsolutions	Produces solution booklet
%  solutionsonly    Solution booklet with questions suppressed
%  specimensolutions SSEP ("specimen,showsolutions" achieves the same)
%  qp-and-ss        Builds QP and solutions in separate
%                   PDFs from single compilation.
% See documentation for other options, usage, etc.
%%%%%%%%%%%%%%%%%%%%%%%%%%%%%%%%%%%%%%%%%%%
% MODULE/DATE/TIME INFORMATION
%%%%%%%%%%%%%%%%%%%%%%%%%%%%%%%%%%%%%%%%%%%
\modulecode{MST124}
\session{C} % Can be omitted if "specimen" or "secondspecimen" options are on
\moduletitle{Title of Module}
%\examcode{MST1241706F1PV1} % Generates barcode
\examtime{10:00\,am -- 1:00\,pm}
\examday{Tuesday 7}
\exammonth{June}
\examyear{2024}
\timeallowed{3 hours}
\copyrightyear{2024} %Optional: same as \examyear if omitted
%\optiontext{} %Can be used to change text appearing at top of multi-choice options.
%\oulogofilebase{OU_Compact_LOGO_BLACK_40mm}% Overrides internal OU logo specification
%\switchmarks % Can be used to place marks to right of comments
%%%%%%%%%%%%%%%%%%%%%%%%%%%%%%%%%%%%%%%%%%%%
% INSTRUCTIONS ON FIRST PAGE
%%%%%%%%%%%%%%%%%%%%%%%%%%%%%%%%%%%%%%%%%%%%
\examinstructions{%
\textbf{Instructions}

Insert module-specific examination instructions here
}% ends \examinstructions
%% alternative way of updating the above information
%\metadataset{
%            module code=MST142,
%            session=D,
%            module title=Title of Module,
%            specimen solutions title=spec title,
%            exam code=MST1241706F1PV1,
%            exam time=10am--2pm,
%            exam day=Wed,
%            exam month=Oct,
%            exam year=2017,
%            time allowed=3.5 hours,
%            copyright year=2017,
%            supp no=3,
%            version no=2.12,
%            faculty=STEM,
%            multiple choice option text=choices,
%            instructions= your instructions here,
%            top padding=0.5cm,
% 			 switch marks, % Does same as \switchmarks
%            }
%%%%%%%%%%%%%%%%%%%%%%%%%%%%%%%%%%%%%%%%%%%
\begin{document}
\maketitle

%%%%%%%%%%%%%%%%%%%%%%%%%%%%%%%%%%%
% Begin writing document below here
%%%%%%%%%%%%%%%%%%%%%%%%%%%%%%%%%%%

\question[optional space for title]
% Questions can be wrapped inside the `questionblock' environment
% This allows you to suppress them when producing solutions, using
% the `solutionsonly' option.
\begin{questionblock}
The first question.\mk{12}	
\end{questionblock}

\begin{solution}
Solutions can be placed in-line, to generate solution files from the same source.\mk{3}

Inside solutions, notes can be added to the marks.\mk[Like this]{4}

You can award decimals, e.g.\ half marks\mk{.5}

Everything gets rendered nicely and added up.\mk{1.5}

You can display a mark but prevent it from getting added to the total and subtotal counters.\mk*[An alternative solution perhaps]{4}[-5pt]

Automatically add up part marks inside the solution environment:
\subtotal*

After a subtotal, you can have some more marks.\mk{3}

You can also manually specify a total or subtotal. Note that these subtotals do not get added to the question total. 
\subtotal{4}

\total*
\end{solution}


\question
\begin{questionblock}
The second question.\mk{3}	
\end{questionblock}

\begin{solution}
Solutions can be placed in-line, to generate solution files from the same source.\mk[Be generous!]{3}
\end{solution}

\question[(example of creating subquestions)]
\begin{enumerate}
\item Use the standard enumerate environment.\mk{3}
\item \begin{enumerate}
\item Sub-sub part.\mk{2}
\item Sub-sub part.\mk{2}
\begin{solution}
Solutions can be created per sub-question, or per sub-sub-question, or after the entire question.\mk{2}

Note that if you insert solutions per sub-question, then you cannot use the `solutionsonly' features.
\end{solution}
\end{enumerate}
\end{enumerate}
The enumitem package allows numbering to be resumed later, to replace the OUTeX `intertext' command.
\begin{enumerate}[resume]
\item Numbering continues from before.\mk{4}
\begin{solution}
Within a question, the total and subtotal commands will remember marks from one solution environment to the next.\mk{2}

\total*
\end{solution}
\end{enumerate}

\question
\emph{This question is about nothing in particular.}
\begin{enumerate}
\item one\mk 4
\item two, with a sufficiently long question that it runs over more than one line.\mk{4}
\item 
\begin{enumerate}
\item Sub-sub\mk 2
\item Sub-sub that is sufficiently long that it runs over more than one line.\mk 2
\end{enumerate}
\item Starred enumerate environment gives you inline enumerate:
\begin{enumerate*}
\item first part
\item second part
\item third part\mk{3}
\end{enumerate*}
\end{enumerate}

\section{Multiple choice questions}


\question
Choose an option.
\begin{options}
\item There are no options
\item There is one option
\item This option is the wrong choice\mk{3}
\end{options}


\question
Choose an inline option.
\begin{inlineoptions}{5} %parameter specifies number of columns
\item yes
\item no
\item maybe
\item well...
\item never!
\item yes
\item yes
\item yes
\noitem %HACK. adds "blank" space so that the items line up as expected
\noitem\mk{3}
\end{inlineoptions}

\question
Choose an inline option.
\begin{inlineoptions}[1.5]{3} %optional parameter specifies line spacing
\item yes
\item no
\item maybe
\item well...
\item never!
\item yes
\item yes
\item yes
\noitem %HACK. adds "blank" space so that the items line up as expected
\end{inlineoptions}

\section{Section title}

The class file is based on the standard article class file, so commands such as section, subsection etc. can all be used if desired.


\question
\emph{This question is about nothing in particular.}
\begin{enumerate}
\item one\mk 4
\item two, with a sufficiently long question that it runs over more than one line.\mk{4}
\item 
\begin{enumerate}
\item Sub-sub\mk 2
\item Sub-sub that is sufficiently long that it runs over more than one line.\mk 2
\end{enumerate}
\end{enumerate}



\question
\emph{This question is about nothing in particular.}
\begin{enumerate}
\item one\mk 4
\item two, with a sufficiently long question that it runs over more than one line.\mk{4}
\item 
\begin{enumerate}
\item Sub-sub\mk 2
\item Sub-sub that is sufficiently long that it runs over more than one line.\mk 2
\end{enumerate}
\end{enumerate}



\question
\emph{This question is about nothing in particular.}
\begin{enumerate}
\item one\mk 4
\item two, with a sufficiently long question that it runs over more than one line.\mk{4}
\item 
\begin{enumerate}
\item Sub-sub\mk 2
\item Sub-sub that is sufficiently long that it runs over more than one line.\mk 2
\end{enumerate}
\end{enumerate}



\question
\emph{This question is about nothing in particular.}
\begin{enumerate}
\item one\mk 4
\item two, with a sufficiently long question that it runs over more than one line.\mk{4}
\item 
\begin{enumerate}
\item Sub-sub\mk 2
\item Sub-sub that is sufficiently long that it runs over more than one line.\mk 2
\end{enumerate}
\end{enumerate}

\question
\emph{This question is about nothing in particular.}
\begin{enumerate}
\item one\mk 4
\item two, with a sufficiently long question that it runs over more than one line.\mk{4}
\item 
\begin{enumerate}
\item Sub-sub\mk 2
\item Sub-sub that is sufficiently long that it runs over more than one line.\mk 2
\end{enumerate}
\end{enumerate}


%\putendtext % Can be used to make the [END OF QUESTION PAPER] text appear before \end{document}

\end{document}

%% Optional section, if you want to insert solutions at the end of the document to form a mark scheme.

\solutions

Another way to generate solutions is to use the solutions switch at the end of the file. These will appear irrespective of the showsolutions option.

\question[optional space for title]
The solution to the first question.\mk{12}

\question
The solution to the second question.\mk{3}


\end{document}

