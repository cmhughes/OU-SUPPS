\documentclass{ouexam} 
% Main available options:
%  specimen			For specimen exam papers
%  secondspecimen	For 2nd specimen exam
%  showsolutions		Produces solution booklet
%  specimensolutions SSEP ("specimen,showsolutions" achieves the same)
% See documentation for other options, usage, etc.
%%%%%%%%%%%%%%%%%%%%%%%%%%%%%%%%%%%%%%%%%%%
% MODULE/DATE/TIME INFORMATION
%%%%%%%%%%%%%%%%%%%%%%%%%%%%%%%%%%%%%%%%%%%
\modulecode{MST124}
\session{C} % Can be omitted if "specimen" or "secondspecimen" options are on
\moduletitle{Title of Module}
\examcode{MST1241606D1PV1} % Generates barcode
\examtime{10:00\,am -- 1:00\,pm}
\examday{Tuesday 7}
\exammonth{October}
\examyear{2016}
\timeallowed{3 hours}
\copyrightyear{2016} %Optional: same as \examyear if omitted
%\optiontext{} %Can be used to change text appearing at top of multi-choice options.
%%%%%%%%%%%%%%%%%%%%%%%%%%%%%%%%%%%%%%%%%%%%
% INSTRUCTIONS ON FIRST PAGE
%%%%%%%%%%%%%%%%%%%%%%%%%%%%%%%%%%%%%%%%%%%%
\examinstructions{%
\textbf{Instructions}

Insert module-specific examination instructions here
}% ends \examinstructions
%%%%%%%%%%%%%%%%%%%%%%%%%%%%%%%%%%%%%%%%%%%
\begin{document}
\maketitle


%%%%%%%%%%%%%%%%%%%%%%%%%%%%%%%%%%%
% Begin writing document below here
%%%%%%%%%%%%%%%%%%%%%%%%%%%%%%%%%%%

\question[optional space for title]
The first question.\mk{12}

\begin{solution}
Solutions can be placed in-line, to generate solution files from the same source.\mk{3}

Inside solutions, notes can be added to the marks.\mk[Like this]{4}
\end{solution}

\question
The second question.\marks{3}

\begin{solution}
Solutions can be placed in-line, to generate solution files from the same source.\mk[Be generous!]{3}
\end{solution}

\question[(example of creating subquestions)]

\begin{enumerate}
\item Use the standard enumerate environment.\mk{3}
\item \begin{enumerate}
\item Sub-sub part.\mk{2}
\item Sub-sub part.\mk{2}
\begin{solution}
Solutions can be created per sub-question, or per sub-sub-question, or after the entire question.\mk{2}
\end{solution}
\end{enumerate}
\end{enumerate}
The enumitem package allows numbering to be resumed later, to replace the OUTeX `intertext' command.
\begin{enumerate}[resume]
\item Numbering continues from before.\mk{4}
\end{enumerate}



\question
\emph{This question is about nothing in particular.}
\begin{enumerate}
\item one\mk 4
\item two, with a sufficiently long question that it runs over more than one line.\mk{4}
\item 
\begin{enumerate}
\item Sub-sub\mk 2
\item Sub-sub that is sufficiently long that it runs over more than one line.\mk 2
\end{enumerate}
\end{enumerate}

\section{Multiple choice questions}


\question
Choose an option.
\begin{options}
\item There are no options
\item There is one option
\item This option is the wrong choice
\end{options}


\question
Choose an inline option.
\begin{inlineoptions}{5} %parameter specifies number of columns
\item yes
\item no
\item maybe
\item well...
\item never!
\item yes
\item yes
\item yes
\noitem %HACK. adds "blank" space so that the items line up as expected
\noitem
\end{inlineoptions}

\question
Choose an inline option.
\begin{inlineoptions}[1.5]{3} %optional parameter specifies line spacing
\item yes
\item no
\item maybe
\item well...
\item never!
\item yes
\item yes
\item yes
\noitem %HACK. adds "blank" space so that the items line up as expected
\end{inlineoptions}

\section{Section title}

The class file is based on the standard article class file, so commands such as section, subsection etc. can all be used if desired.


\question
\emph{This question is about nothing in particular.}
\begin{enumerate}
\item one\mk 4
\item two, with a sufficiently long question that it runs over more than one line.\mk{4}
\item 
\begin{enumerate}
\item Sub-sub\mk 2
\item Sub-sub that is sufficiently long that it runs over more than one line.\mk 2
\end{enumerate}
\end{enumerate}



\question
\emph{This question is about nothing in particular.}
\begin{enumerate}
\item one\mk 4
\item two, with a sufficiently long question that it runs over more than one line.\mk{4}
\item 
\begin{enumerate}
\item Sub-sub\mk 2
\item Sub-sub that is sufficiently long that it runs over more than one line.\mk 2
\end{enumerate}
\end{enumerate}



\question
\emph{This question is about nothing in particular.}
\begin{enumerate}
\item one\mk 4
\item two, with a sufficiently long question that it runs over more than one line.\mk{4}
\item 
\begin{enumerate}
\item Sub-sub\mk 2
\item Sub-sub that is sufficiently long that it runs over more than one line.\mk 2
\end{enumerate}
\end{enumerate}



\question
\emph{This question is about nothing in particular.}
\begin{enumerate}
\item one\mk 4
\item two, with a sufficiently long question that it runs over more than one line.\mk{4}
\item 
\begin{enumerate}
\item Sub-sub\mk 2
\item Sub-sub that is sufficiently long that it runs over more than one line.\mk 2
\end{enumerate}
\end{enumerate}

\question
\emph{This question is about nothing in particular.}
\begin{enumerate}
\item one\mk 4
\item two, with a sufficiently long question that it runs over more than one line.\mk{4}
\item 
\begin{enumerate}
\item Sub-sub\mk 2
\item Sub-sub that is sufficiently long that it runs over more than one line.\mk 2
\end{enumerate}
\end{enumerate}


%\putendtext % Can be used to make the [END OF QUESTION PAPER] text appear before \end{document}

\end{document}

%% Optional section, if you want to insert solutions at the end of the document to form a mark scheme.

\solutions

Another way to generate solutions is to use the solutions switch at the end of the file. These will appear irrespective of the showsolutions option.

\question[optional space for title]
The solution to the first question.\mk{12}

\question
The solution to the second question.\mk{3}


\end{document}

